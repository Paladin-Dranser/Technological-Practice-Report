\sectionWithoutNumber{Заключэнне}

Асноўнай задачай на час праходжання практыкі было прямыненне ведаў, які атрыманыя ў час навучання ў акадэміі,
у працоўнай дзейнасці і атрыманне практычнага досведу.

У першы дзень наведвання практыкі быў складзены графік працы, у адпаведнасці з якім выконваліся заданні, выдадзеныя кіраўніком практыкі ад прадпрыемства, а таксама праведзены інструктаж па тэхніцы бяспекі. Усе дадзеныя заносіліся ў дзённік.

Практычнае прымяненне тэарэтычнай базы дазваляе стварыць неабходны для працы рэсурс досведу, праявіць здольнасці ў абранай спецыяльнасці і зарыентавацца ў прафесіі. Вытворчая практыка спрыяе станаўленню студэнта як кваліфікаванага спецыяліста, дапамагаючы развіць прафесійныя і асабістыя якасці, развіваючы навыкі працы ў калектыве і самастойна.
