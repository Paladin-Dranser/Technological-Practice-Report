% Раздзел 1 — метадалогія devops
\section{Кароткая характарыстыка прадпрыемтсва}

\subsection{Гісторыя развіцця прадпрыемства}
«Белтэлекам» быў створаны 3 ліпеня 1995 года как Рэспубліканскае дзяржаўнае аб'яднанне.
1 жніўня 2004 года кампанія прайшла пераўтварэнне ў Рэспубліканскае ўнітарнае прадпрыемства электрасувязі.
Абласныя ўнітарныя прадпремствы электрасувязі, а таксама УП «Мінская гарадская тэлефонная сетка»,
УП «Міжгародняя сувязь», УП «Мінская тэлефонна-тэлеграфная станцыя» былі рэарганізаваны ў філіялы РУП «Белтэлекам» шляхам далучэння.

Лідскі занальны вузел электрасувязі  (ЗВЭС) Гродзенскага філіяла <<Белтэлекам>>
быў заснаваны ў 1996 годзе і прадастаўляе паслугі электрасувязі ў горадзе Ліда і лідскім раёне.

\subsection{Агульны эканамічны і тэхнічны стан}
Брэнд-партфель кампаніі складаецца з гандлёвай маркі «Белтэлекам»,
якім прадстаўлена галасавая сувязь, перадача даных, хостынг і шэраг іншых паслуг,
а таксама брэндаў byfly (высокахуткасны доступ у Інтэрнэт) і ZALA (інтэрактыўнае і эфірнае тэлебачанне).

Колькасць абанентаў сеткі «Белтэлекам» у цяперашні час расце.
Па шчыльнасці пранікнення шырокапалоснага доступу ў сетку Інтэрнэт на 100 жыхароў
Рэспубліка Беларусь выйшла на сярэднееўрапейскія паказчыкі.
У 2015 г. абаненцкая база карыстальнікаў шырокапалоснага доступу ў сетку інтэрнэт byfly
перасягнула мяжу ў 2,2 мільёна, а IPTV ZALA наблізілася да 1,2 мільёна.

Паслугі тэлефоннай сувязі дазваляюць у аўтаматычным рэжыме
ажыццяўляць тэлефонныя злучэнні паміж абанентамі.
База стацыянарнай тэлефоннай сувязі ў цэлым па рэспубліцы на 1 студзеня 2016 года ---
больш 4,4 мільёна.
У рэйтынгу Міжнароднага звязу электрасувязі па выніках 2013 года
шчыльнасць тэлефонных апаратаў на 100 чалавек дасягнула 47,8.

Для абслугоўвання кліентаў «Белтэлекам» ўтрымлівае сетку сэрвісных цэнтраў
на тэрыторыі ўсёй краіны.
У сталіцы, абласных і раённых цэнтрах працуе 143 сэрвісных цэнтры кампаніі.
І сёння, дзякуючы намаганням кампаніі «Белтэлекам»,
сучасныя тэлекамунікацыйныя паслугі даступны на тэрыторыі ўсёй краіны.

Кампанія прадастаўляе шырокі спектр паслуг,
які бесперапынна павялічваецца.
Адбываецца пераход з аналагавага абаненцкага доступу на IMS платформу з тэхналогіяй GPON,
што мае на ўвазе замену існуючых медных абаненцкіх ліній на ВАЛС і памяншэнне карыстальнікаў ADSL.

У цяперашні час РУП «Белтэлекам» забяспечвае перадачу тэлевізійных
і радыёвяшчальных праграм па каналах сувязі ад апаратна-студыйных комплексаў (АСК) тэлерадыёкампаній
да радыётэлевізійных перадаючых станцый (РТПС).
Па заяўках тэлерадыёкампаній арганізоўваецца падача пазапланавых тэлевізійных перадач.
Да ўсіх дзеючых аналагавых і лічбавых перадатчыкаў арганізавана падача тэле- ці радыё-каналаў, як у аналагавым, так і ў лічбавым фармаце.

У рамках выканання «Дзяржаўнай праграмы ўкаранення лічбавага тэлевізійнага і радыёвяшчання ў Рэспубліцы Беларусь да 2015 года»
было прадугледжана будаўніцтва сеткі наземнага лічбавага эфірнага тэлебачання
і арганізацыя лічбавага эфірнага тэлерадыёвяшчання на 93 радыётэлевізійных перадаючых станцыях.
РУП «Белтэлекам» з 2005 года ажыццяўляе планавую мадэрнізацыю сеткі распаўсюду тэлевізійных праграм
і будаўніцтва валаконна-аптычных ліній сувязі.
Пры гэтым прымяняецца сучаснае абсталяванне, якое ажыццяўляе перадачу сігналу тэлеканалаў
у лічбавым фармаце да лічбавых эфірных перадатчыкаў, устаноўленых на радыётэлевізійных перадаючых станцыях.
У 2015 годзе РУП «Белтэлекам» скончыў пракладку валаконна-аптычнага кабеля і падачу лічбавага пакета тэлевізійных праграм
да ўсіх радыётэлевізійных перадаючых станцый рэспублікі, якія плануюцца да будаўніцтва.

\subsection{Структура базавай арганізацыі}
Пад арганізацыйнай структурай кіравання неабходна разумець сукупнасць кіраўнічых звёнаў,
размешчаных у строгай супадпарадкаванасці і якія забяспечваюць ўзаемасувязь паміж кіруючай і кіраванай сістэмамі.

У цяперашні час у склад унітарнага прадпрыемства ўваходзяць наступныя філіялы:
\begin{enumerate}
    \item філіял "Мінская гарадская тэлефонная сетка» (МГТС);
    \item філіял "Міжгародняя сувязь» (ГП «Міжгародняя сувязь»);
    \item філіял "Мінская тэлефонна-тэлеграфная станцыя» (МТТС);
    \item Брэсцкі філіял;
    \item Віцебскі філіял;
    \item Гомельскі філіял;
    \item Гродзенскі філіял;
    \item Мінскі філіял;
    \item Магілёўскі філіял;
    \item Дапаможная сельская гаспадарка.
\end{enumerate}

У склад РУП ўваходзіць галаўное падраздзяленне, якое ўключае:
\begin{enumerate}
    \item апарат кіравання;
    \item вытворчасць «Міжнародны цэнтр камутацыі»;
    \item вытворчасць «Інфармацыйна-разліковы цэнтр»;
    \item сталовая;
    \item цэнтральны склад;
    \item аздараўленчы цэнтр «Загор'е»;
    \item музей сувязі.
\end{enumerate}

У склад РУП «Белтэлекам» уваходзіць упраўленне ўзаемадзеяння з аператарамі,
якое ажыццяўляе арганізацыю і ўзаемаразлікі па міжнародных паслугах электрасувязі.

На ўнітарным прадпрыемстве ствараецца і дзейнічае ў якасці калектыўнага органа кіравання
Савет Унітарнага прадпрыемства ў колькасці 12 (дванаццаці) чалавек,
які складаецца з роўнай колькасці прадстаўнікоў,
прызначаных генеральным дырэктарам унітарнага прадпрыемства і прадстаўнікоў, абраных працоўным калектывам.

У кампетэнцыю Савета унітарнага прадпрыемства ўваходзіць:
\begin{enumerate}
    \item вырашэнне сацыяльна-эканамічных задач;
    \item вызначэнне парадку размеркавання чыстага прыбытку;
    \item разгляд канфліктных сітуацый,
          якія ўзнікаюць паміж адміністрацыяй унітарнага прадпрыемства і працоўным калектывам,
          і прыняцце мер па іх вырашэнні.
\end{enumerate}

Старшыня Савета унітарнага прадпрыемства абіраецца на пасяджэнні
Савета унітарнага прадпрыемства адкрытым або тайным галасаваннем з ліку яго членаў.

Пасяджэння Савета унітарнага прадпрыемства праводзяцца па меры неабходнасці,
але не менш за два разы на год. Яго пасяджэнні з'яўляюцца правамоцнымі,
калі на іх прысутнічае не менш за 2/3 яго членаў.

\subsection{Палітыка РУП <<Белтэлекам>> у галіне якасці}
Стратэгічная мэта РУП «Белтэлекам» --- забеспячэнне доўгатэрміновай канкурэнтаздольнасці
і павышэнне эканамічнай эфектыўнасці кампаніі шляхам своечасовага
і поўнага задавальнення попыту на сучасныя і даступныя паслугі электрасувязі
Для дасягнення стратэгічнай мэты РУП «Белтэлекам» ставіць перад сабой наступныя задачы:
\begin{enumerate}
    \item якасна аказваць паслугі электрасувязі
    \item пашыраць прысутнасць кампаніі на рынку за кошт укаранення новых паслуг, прапаноўваць зручныя для спажыўцоў тарыфныя планы, прымяняць эфектыўныя рынкавыя стратэгіі
    \item бесперапынна паляпшаць вытворчую базу кампаніі шляхам укаранення тэхналагічных інавацый і ўдасканалення бізнес-працэсаў
    \item развіваць і замацоўваць кадравы патэнцыял кампаніі
\end{enumerate}

Прынцыпы выканання пастаўленых задач:
\begin{enumerate}
    \item прымяненне інструментаў эфектыўнага менеджменту, у першую чаргу - сістэмы менеджменту якасці, працэснага падыходу з мышленнем на аснове рызык
    \item адпаведнасць патрабаванням і пастаяннае павышэнне выніковасці сістэмы менеджменту якасці
    \item арыентацыя на спажыўцоў, выкананне патрабаванняў спажыўцоў і павышэнне іх задаволенасці
    \item стварэнне высокапрадукцыйных працоўных месцаў, развіццё кампетэнцый і забеспячэнне высокага ўзроўню прафесіяналізму персаналу кампаніі
    \item лідэрства кіраўнікоў усіх узроўняў, прыцягненне персаналу да працэсаў павышэння якасці, дакладанае размеркаванне адказнасці і паўнамоцтваў
    \item прызнанне асабістых заслуг супрацоўнікаў, узнагароджанне за ўклад у стварэнне паслуг высокай якасці і ўдасканаленне сістэмы менеджменту якасці
    \item узаемавыгаднае доўгатэрміновае супрацоўніцтва з партнёрамі кампаніі на нацыянальным і міжнародным узроўнях
\end{enumerate}

Вышэйшае кіраўніцтва РУП «Белтэлекам» прымае абавязацельства няўхільна прытрымлівацца выкладзеных прынцыпаў, бярэ на сябе адказнасць за забеспячэнне эфектыўнага кіравання прадпрыемствам, стварэнне неабходных арганізацыйных умоў, выдзяленне фінансавых, кадравых і матэрыяльных рэсурсаў.
