\section{Праграмнае забеспячэнне на прадпрыемствe}

\subsection{NGINX}
NGINX --- веб-сервер і паштовы проксі-сервер, які працуе на Unix-падобных аперацыйных сістэмах.

NGINX пазіцыянуецца вытворцам як просты, хуткі і надзейны сервер, не перагружаны функцыямі.

Прымяненне NGINX мэтазгодна перш за ўсё для статычных вэб-сайтаў і як зваротнага проксі-сервера перад дынамічнымі сайтамі.

NGINX у якасці HTTP-сервера прадастаўляе:
\begin{enumerate}
    \item абслугоўванне нязменных запытаў, індэксных файлаў, аўтаматычнае стварэнне спісу файлаў, кэш дэскрыптараў адкрытых файлаў;
    \item паскарэнне праксавання без кэшавання, простае размеркаванне нагрузкі і адмоваўстойлівасць;
    \item модульнасць, фільтры, у тым ліку сціск (gzip), byte-ranges (дапампоўку), chunked адказы, HTTP-аўтэнтыфікацыя, SSI-фільтр;
    \item некалькі подзапросов на адной старонцы, апрацоўваныя ў SSI-фільтры праз проксі або FastCGI, выконваюцца паралельна;
    \item падтрымка SSL;
    \item падтрымка PSGI, WSGI;
    \item эксперыментальная падтрымка убудаванага Perl.
\end{enumerate}

\subsection{Apache HTTP Server}
Асноўнымі перавагамі Apache лічацца надзейнасць і гнуткасць канфігурацыі. Ён дазваляе
падключаць вонкавыя модулі для прадастаўлення даных,
выкарыстоўваць СКБД для аўтэнтыфікацыі карыстальнікаў, мадыфікаваць паведамленні пра памылкі і да т. п..
Падтрымлівае IPv6.

\subsection{Apache Tomcat}
Tomcat --- кантэйнер сeрвлета з адкрытым зыходным кодам, які распрацоўваецца Apache Software Foundation. Рэалізуе спецыфікацыю сeрвлета, спецыфікацыю JavaServer Pages (JSP) і JavaServer Faces (JSF). Напісаны на мове Java.

Tomcat дазваляе запускаць вeб-праграмы і змяшчае шэраг праграм для самастойнай настройкі.

Tomcat выкарыстоўваецца ў якасці самастойнага вeб-сервера, у якасці сервера кантэнту ў спалучэнні з вeб-серверам Apache HTTP Server, а таксама ў якасці кантэйнера сeрвлета ў серверах праграм JBoss і GlassFish.

\subsection{Zabbix}
Zabbix --- свабодная сістэма маніторынгу і адсочвання стану разнастайных сервісаў камп'ютарнай сеткі, сервераў і сеткавага абсталявання.

Для захоўвання дадзеных выкарыстоўваецца MySQL, PostgreSQL, SQLite або Oracle Database, вeб-інтэрфейс напісаны на PHP.

Падтрымлівае некалькі відаў маніторынгу:
\begin{enumerate}
    \item Simple checks --- можа правяраць даступнасць і рэакцыю стандартных сeрвісаў, такіх як SMTP або HTTP, без ўстаноўкі якога-небудзь праграмнага забеспячэння на назіраным хасце.
    \item Zabbix agent --- можа быць усталяваны на UNIX-падобных або Windows-хастах для атрымання дадзеных пра нагрузку працэсара, выкарыстання сеткі, дыскавай прасторы і гэтак далей.
    \item External check --- выкананне знешніх праграм, таксама падтрымліваецца маніторынг праз SNMP.
\end{enumerate}
