\section{Распрацоўка аўтаматычнага разгортвання інфраструктуры для веб-сервісаў}

\subsection{Кароткае апісанне інфраструктуры}

\subsection{Інфраструктура як код}

\subsubsection{Канфігурацыйны файл разгортвання інфраструктуры.}
Канфігурацыйны файл разгортвання інфраструктуры прадстаўлены ў лістынку
\ref{lst:Vagrantfile}:

\lstinputlisting[caption={Зыходны код для разгортвання інфраструктуры},%
                            label={lst:Vagrantfile},%
                            language=Ruby]{Vagrantfile.rb}

У дадзенай канфігурацыі прадстаўлена апісанне 2 віртуальных машын,
Zabbix-Server і Zabbix-Agent.

Zabbix-Server --- віртуальная машына, на якой запушчаны Zabbix для
маніторынгу іншых сервісаў у сетцы.

Zabbix-Agent --- віртуальная машына, на якой запушчаны які-небудзь сервіс,
у дадзеным выпадку Tomcat, для абслугоўвання кліентаў.

Настройкі для Zabbix-Server і Zabbix-Agent прадастаўляюцца пры дапамозе
bash-скрыптаў zabbix-server.sh і zabbix-agent.sh адпаведна.
Канфігурацыйныя файлы настройкі будуць разгледжаны ў наступных раздзелах.

\subsubsection{Канфігурацыйны файл настройкі Zabbix-сервера.}
Канфігурацыйны файл настройкі Zabbix-сервера прадстаўлены ў лістынгу
\ref{lst:Zabbix-Server}:

\lstinputlisting[caption={Зыходны код для настройкі Zabbix сервера},%
                            label={lst:Zabbix-Server},%
                            language=Bash]{zabbix-server.sh}

Дадзены bash-скрыпт падчас разгортвання віртуальнай машыны
ўстанаўлівае і канфігурыруе для іх узаемадзеяння
базу даных MariaDB, Zabbix сервер, Apache сервер.

\subsubsection{Канфігурацыйны файл настройкі веб-сервера.}
Канфігурацыйны файл настройкі веб-сервера прадстаўлены ў лістынгу
\ref{lst:Zabbix-Agent}:

\lstinputlisting[caption={Зыходны код для настройкі веб-сервера},%
                            label={lst:Zabbix-Agent},%
                            language=Bash]{zabbix-agent.sh}

Дадзены bash-скрыпт падчас разгортвання віртуальнай машыны
ўстанаўлівае і канфігурыруе NGINX у якасці reverse-proxy сервера,
Tomcat для прадастаўлення веб-праграм кліенту, які "схаваны" за NGINX серверам.

У той жа час адбываецца настройка Tomcat сервіса для магчымасці адсочваць
java-параметры з удалённага хаста.

Пасля ўстаноўкі ўсіх неабходных кампанентаў запускаецца python-скрыпт
<<zabbix-apy.py>>, які падключае дадзеную віртуальную машыну да
Zabbix сервера ў прыватнай сетцы.

\subsubsection{Python-праграма для аўтаматычнай рэгістрацыі Zabbix-агента.}
\lstinputlisting[caption={Зыходны код для аўтаматычнай рэгістрацыі Zabbix-агента},%
                            label={lst:Python},%
                            language=Python]{zabbix-api.py}
