\section{Распрацоўка аўтаматычнага разгортвання інфраструктуры для веб-сервісаў}

\subsection{Кароткае апісанне інфраструктуры}
На малюнку \ref{figure:diagram} прадстаўлена спрошчаная дыяграма інфраструктуры
для прадастаўлення кліентам веб-паслуг.

\begin{figure}[h!]
    \centering
    \includegraphics{diagram.pdf}
    \caption{Дыяграма інфраструктуры для веб-сервісаў}
    \label{figure:diagram} 
\end{figure}

Як бачна з дыяграмы, сервер з Tomcat не будзе напрамую даступны з Інтэрнэта.

NGINX сервер выконвае ролі балансірошчыка нагрузкі і reverse-proxy сервера.
NGINX сервер прымае запыт кліента з Інтэрнэта і перанакіроўвае яго на
адзін са схаваных Tomcat сервераў (у нашым выпадку ёсць толькі 1 Tomcat сервер).

Zabbix сервер мае доступ з Інтэрнэту для магчымасці адміністравання.
У той жа час ён збірае інфармацыю з усіх Zabbix агентаў, якія знаходзяцца ў
прыватнай сетцы (у нашым выпадку Tomcat сервер).

\subsection{Інфраструктура як код}
Мадэль «Інфраструктура як код (IaC)», якую часам называюць «праграмуемай інфраструктурай», гэта мадэль, па якой працэс настройкі інфраструктуры аналагічны працэсу праграмавання праграмнага забеспячэння.

Інфраструктура як код дазваляе кіраваць віртуальнымі машынамі на праграмным узроўні. Гэта выключае неабходнасць ручной настройкі і абнаўленняў для асобных кампанентаў абсталявання. Інфраструктура становіцца адаптаванай да маштабавання. Адзін аператар можа выконваць разгортванне і кіраванне як адной, так і 1000 машынамі, выкарыстоўваючы адзін і той жа набор кода. Сярод гарантаваных пераваг інфраструктуры як кода - хуткасць, эканамічнасць і памяншэнне рызыкі.

\subsubsection{Канфігурацыйны файл разгортвання інфраструктуры.}
Канфігурацыйны файл разгортвання інфраструктуры прадстаўлены ў лістынку
\ref{lst:Vagrantfile}:

\lstinputlisting[caption={Зыходны код для разгортвання інфраструктуры},%
                            label={lst:Vagrantfile},%
                            language=Ruby]{Vagrantfile.rb}

У дадзенай канфігурацыі прадстаўлена апісанне 2 віртуальных машын,
Zabbix-Server і Zabbix-Agent.

Zabbix-Server --- віртуальная машына, на якой запушчаны Zabbix для
маніторынгу іншых сервісаў у сетцы.

Zabbix-Agent --- віртуальная машына, на якой запушчаны які-небудзь сервіс,
у дадзеным выпадку Tomcat, для абслугоўвання кліентаў.

Настройкі для Zabbix-Server і Zabbix-Agent прадастаўляюцца пры дапамозе
bash-скрыптаў zabbix-server.sh і zabbix-agent.sh адпаведна.
Канфігурацыйныя файлы настройкі будуць разгледжаны ў наступных раздзелах.

\subsubsection{Канфігурацыйны файл настройкі Zabbix-сервера.}
Канфігурацыйны файл настройкі Zabbix-сервера прадстаўлены ў лістынгу
\ref{lst:Zabbix-Server}:

\lstinputlisting[caption={Зыходны код для настройкі Zabbix сервера},%
                            label={lst:Zabbix-Server},%
                            language=Bash]{zabbix-server.sh}

Дадзены bash-скрыпт падчас разгортвання віртуальнай машыны
ўстанаўлівае і канфігурыруе для іх узаемадзеяння
базу даных MariaDB, Zabbix сервер, Apache сервер.

\subsubsection{Канфігурацыйны файл настройкі веб-сервера.}
Канфігурацыйны файл настройкі веб-сервера прадстаўлены ў лістынгу
\ref{lst:Zabbix-Agent}:

\lstinputlisting[caption={Зыходны код для настройкі веб-сервера},%
                            label={lst:Zabbix-Agent},%
                            language=Bash]{zabbix-agent.sh}

Дадзены bash-скрыпт падчас разгортвання віртуальнай машыны
ўстанаўлівае і канфігурыруе NGINX у якасці reverse-proxy сервера,
Tomcat для прадастаўлення веб-праграм кліенту, які "схаваны" за NGINX серверам.

У той жа час адбываецца настройка Tomcat сервіса для магчымасці адсочваць
java-параметры з удалённага хаста.

Пасля ўстаноўкі ўсіх неабходных кампанентаў запускаецца праграма 
<<zabbix-apy.py>>, які падключае дадзеную віртуальную машыну да
Zabbix сервера ў прыватнай сетцы.

\subsubsection{Python-праграма для аўтаматычнай рэгістрацыі Zabbix-агента.}
Python-праграма для аўтаматычнай рэгістрацыі Zabbix-агента на Zabbix серверы ў
прыватнай сетцы прадстаўлена ў лістынгу
\ref{lst:Python}:

\lstinputlisting[caption={Зыходны код для аўтаматычнай рэгістрацыі Zabbix-агента},%
                            label={lst:Python},%
                            language=Python]{zabbix-api.py}

Для змянення параметраў Zabbix сервера выкарыстоўваецца Zabbix API.

У дадзенай праграме выкарыстоўваюцца наступныя функцыі:
\begin{enumerate}
    \item post() -- другасная функцыя, якая выконвае запыт да Zabbix сервера і вяртае вынік аперацыі;
    \item is\_existed\_hostgroup() -- правярае ці існуе перададзеная група на Zabbix серверы;
    \item create\_hostgroup() -- стварае перададзеную групу на Zabbix серверы;
    \item get\_group\_id() --  вяртае id групы па імені групы;
    \item get\_ip\_address\_from\_server\_network() -- знаходзіць лакальны ip адрас машыны ў сетцы з Zabbix серверам;
    \item get\_template\_id() --  вяртае id шаблона па імені шаблона;
    \item is\_existed\_host() -- правярае ці існуе перададзеная імя хаста на Zabbix серверы;
    \item create\_host() -- рэгістрыруе машыну ў Zabbix серверы.
\end{enumerate}
