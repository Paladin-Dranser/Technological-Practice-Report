%Прадмова
\sectionWithoutNumber{\prefacename}
Рэспубліканскае ўнітарнае прадпрыемства электрасувязі «Белтэлекам» ---
вядучая тэлекамунікацыйная кампанія з шматгадовай гісторыяй,
персанал якой забяспечвае і развівае важныя для дзяржавы, грамадства,
прыватных і карпаратыўных кліентаў тэхналогіі сувязі.
У сваёй дзейнасці прадпрыемства робіць стаўку на актыўную палітыку пашырэння
і паляпшэння паслуг электрасувязі.
Кампанія дынамічна развіваецца і займае дамінуючую пазіцыю на
тэлекамунікацыйным рынку Рэспублікі Беларусь,
з'яўляючыся найбуйнейшым аператарам электрасувязі на тэрыторыі краіны.

«Белтэлекам» быў створаны 3 ліпеня 1995 года як Рэспубліканскае дзяржаўнае аб'яднанне.
1 жніўня 2004 года кампанія была ператворана ў
Рэспубліканскае ўнітарнае прадпрыемства электрасувязі.
Абласныя унітарныя прадпрыемствы электрасувязі,
а таксама УП «Мінская гарадская тэлефонная сетка»,
УП «Міжгародняя сувязь», УП «Мінская тэлефонна-тэлеграфная станцыя»
былі рэарганізаваны ў філіялы РУП «Белтэлекам» шляхам далучэння.

Сёння кампанія «Белтэлекам» уключае ў сябе 9 філіялаў
і 3 вытворчасці ў складзе галаўнога структурнага падраздзялення прадпрыемства.

Місія кампаніі «Белтэлекам» - аб'ядноўваць людзей, даючы свабоду зносін і атрымання інфармацыі.

Асноўнай задачай усіх структурных падраздзяленняў РУП «Белтэлекам»
з'яўляецца своечасовае, якаснае і поўнае задавальненне патрэб паслугамі
электрасувязі юрыдычных і фізічных асоб.

Адным з важных паказчыкаў эксплуатацыі сетак электрасувязі РУП «Белтэлекам»
з'яўляецца аператыўнае і якаснае выпраўленне пашкоджанняў.

Правільны ўлік заяў і выпраўленых пашкоджанняў,
працягласць ліквідацыі пашкоджанняў характарызуе эксплуатацыйна-тэхнічны стан сетак,
дае магчымасць праводзіць аналіз і
на яго аснове прымаць неабходныя меры для паляпшэнню якасці іх эксплуатацыі.

\clearpage
